\chapter{Methodology}
\label{chap:method}

\section{Writing text}
\label{sec:meth/text}

Text can be written with:
\begin{itemize}
    \item \texttt{typewriter} font
    \item \textsf{sans serif} font
    \item \textit{italic} font
    \item \textbf{boldface} font
    \item \emph{emphasis} font
\end{itemize}

That is a unnumbered list. You can also write a numbered list:
\begin{enumerate}
    \item First item
    \item Second item
    \item Third item
\end{enumerate}

You can also write a description list:
\begin{description}
    \item[Accumulated input stream] Would provide inconsistent ground truth across \break frames for terrain that is newly explored
    \item[Initially mapped out environments] This would involve the need for a wide variety of terrains (large datasets available) or a lot of manual work in mapping out terrains.
\end{description}

\subsection{Introduction}
\label{ssec:meth/text/intro}

\subsubsection{Subsubsection}

\paragraph{Paragraph}
Citations can be made as \citep{Heisenberg1983} or \citet{Heisenberg1983}. Multiple citations can be made together as \citep{Heisenberg1983,Einstein1905} Links to webpages as \url{https://www.google.com} or \href{https://www.google.com}{Google}. 

Links to figures, tables and sections can be made using \cref{fig:network_archs}, \cref{tab:terrain_params} and \cref{ssec:meth/text/intro} respectively. You can also refer to equations as \cref{eq:example_equation}. \Cref{tab:terrain_params} can be used at the beginning of a sentence. 

Math inline can be written as $x^2 + y^2 = z^2$ or in display mode as
\begin{equation}
    x^2 + y^2 = z^2
    \label{eq:example_equation}
\end{equation}
or in a code block as
\begin{verbatim}
x^2 + y^2 = z^2
\end{verbatim}

\begin{table}
    \centering
    \begin{tabular}{|c|c|}
        \hline
        \textbf{Terrain Part} & \textbf{Specification} \\ \hline
        Wall & height=$3.0$ m \\
        Platforms & of heights $ \in \left\{0.0, 0.5, 1.0, 1.5, 2.0\right\} $\\
        Stairs & within heights $ \in \left\{0.0, 0.5, 1.0, 1.5, 2.0\right\} $\\
        Ramps & within heights $ \in \left\{0.0, 0.5, 1.0, 1.5, 2.0\right\} $\\
        \hline
    \end{tabular}
    \caption{Specifications of terrain parts}
    \label{tab:terrain_params}
\end{table}

\begin{figure}[btph]
    \centering
    \begin{subfigure}[b]{0.47\linewidth}
        \centering
        \includegraphics[width=0.8\linewidth]{example-image-a}
        \caption{Convolutional Autoencoder (CAE)}
        \label{fig:conv_arch}
    \end{subfigure}
    \hfill
    \begin{subfigure}[b]{0.47\linewidth}
        \centering
        \includegraphics[width=0.8\linewidth]{example-image-b}
        \caption{CAE with Memory}
        \label{fig:convmem_arch}
    \end{subfigure}
    \\
    \begin{subfigure}[b]{0.47\linewidth}
        \centering
        \includegraphics[width=0.8\linewidth]{example-image-c}
        \caption{CAE with Transformed Output}
        \label{fig:davidu_arch}
    \end{subfigure}
    \hfill
    \begin{subfigure}[b]{0.47\linewidth}
        \centering
        \includegraphics[width=0.8\linewidth]{example-image}
        \caption{CAE with Transformed Latent}
        \label{fig:convmemtr_arch}
    \end{subfigure}
    \caption{Network architectures considered}
    \label{fig:network_archs}
\end{figure}
